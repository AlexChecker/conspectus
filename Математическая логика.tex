\documentclass[]{article}
\usepackage[utf8x]{inputenc}
\usepackage[russian]{babel}
\usepackage{hyperref}
\usepackage{amsmath}
\usepackage{amssymb}
\usepackage{cancel}
\title{Математическая логика}
\author{Александр Голованов}

\begin{document}
\maketitle
\newpage
\tableofcontents
\newpage

\section{17 февраля 2025}
\subsection{Графы}
Граф называется связанным, если каждые его две вершины связаны

\subsubsection{Задача 1}

\textbf{Условие: }Из пункта А в пункт Б выехали 5 машин разного цвета: белая, черная, красная,синяя, зеленая. Черная едет впереди синей, зеленая впереди белой но позади синей. Красная едет впереди черной

\subsection{Ориентированные графы}

Одна и та же вершина графа может быть как началом ребра, так и его концом. Ребро, которое возвращается в точку старта называется петлей

\subsection{Способы задания графов}
\begin{enumerate}
\item Аналитический
\item Геометрический
\item Матричный
\end{enumerate}

Матрица, элементами которой являются только 0 и единицы и некое число m называется матрицей смежности графа G. m - число ребер графа, идущих из одной вершины в другую.

\begin{gather*}
\begin{vmatrix}
0 & 3 & 0& 0 & 0&1&0\\
3  & 0 & 1 & 0 & 0 &0& 0\\
0 & 1&1&0&1&0&1\\
0&0&1&0&1&1&0\\
1&0&0&0&1&0&1\\
0&0&1&1&2&0&0\\
\end{vmatrix}
\end{gather*}

\end{document}