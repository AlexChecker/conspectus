\documentclass[]{article}
\usepackage[utf8x]{inputenc}
\usepackage[russian]{babel}
\usepackage{hyperref}
\usepackage{amsmath}
\usepackage{amssymb}
\usepackage{cancel}
\title{Базы данных}
\author{Александр Голованов}

\begin{document}
\maketitle
\newpage
\tableofcontents
\newpage

\section{18 февраля 2025}
\subsection{Основные определения}
Таблица - основная единица хранения данных. Таблицы содержат строки(записи) и столбцы(атрибуты). Каждая таблица представляет собой сущность (объект) реального мира.
\newline
\textbf{Строка, запись или кортеж} - один экземпляр сущности
\newline
\textbf{Столбец или атрибут} представляет собой характеристику сущности
\newline
\textbf{Домен} - множество допустимых значений для атрибута
\newline
\textbf{Ключ} - один или несколько атрибутов, которые однозначно идентифицируют строку в таблице
\newline 
 \subsubsection{Виды ключей}: 
\begin{enumerate}
\item \textbf{Первичный ключ} - уникально идентифицирует каждую строку в таблице. Каждая таблица должна иметь только один первичный ключ. Первичный ключ не может содержать пустое значение (NULL).
\item \textbf{Вторичный (внешний) ключ} - атрибут или группа атрибутов в одной таблице, который ссылается на первичный ключ в другой таблице
\item \textbf{Составной ключ} - первичный ключ, состоящий из нескольких атрибутов
\end{enumerate}
\subsection{Схема базы данных}
\textbf{Схема базы данных} - логическая структура БД, которая определяет таблицы, атрибуты, типы данных, ключи и связи между таблицами. 

\subsection{Отношения}
\textbf{Отношение} описывает связь между двумя или более таблицами. 
Виды отношений:
\begin{enumerate}
\item Один ко многим (1КМ) - одна запись в таблице A может быть связана с несколькими записями в таблице B.
\item Один к одному (1К1) - одна запись в таблице A связана только с одной записью в таблице B.
\item Многие ко многим (МКМ) - множество записей в таблице A могут быть связаны со множеством записей в таблице B. 
\end{enumerate}
\end{document}